\documentclass[a4paper,10pt]{article}
\usepackage[utf8]{inputenc}
\usepackage{url}
\usepackage{amsthm, amscd, amsfonts, amssymb, graphicx,tikz, color, environ}

%opening
\title{Literature Review: Evaluating Different Fault Detection Methods against Simulating Flexible Assembly System Event Logs – DRAFT}
\author{Tero Keski-Valkama}

\begin{document}

\maketitle

\smallskip
\noindent \textbf{Keywords.} Anomaly detection, Flexible Assembly System, Fault detection, Long short-term memory

\section{Topic}
Flexible Assembly System is a modular and reconfigurable assembly and tooling workshop with a focus on small and medium-sized batches of varying products.
Flexible Assembly System was described formally by Donath and Graves \cite{donath1988flexible} as a system consisting of a set of products each with a specified volume
assembled on a workshop consisting of a fixed number of cells.

The assembly steps are performed in cells in parallel. Assemblies and components are transported to the cell where they are combined, tooled or inspected. Products
are transported out of the cell for the next assembly step in another cell, or out of the system as end products.
The work steps executed in the cells can be manual or automatic. There can be a central storage such as a shelf for storing components, the intermediate assemblies between the steps
and the end products waiting to be transported out of the system.

Flexible Assembly System work loads consist of small and medium sized batches where there can be some variation in the products based on customization and personalization.
The Flexible Assembly Systems are modular and often composed out of independent modules from different suppliers. The Flexible Assembly System event stream consists
of events received from all the separate modules of the system, and additional sensors and triggers added to the system in integration phase or later.

Using a Flexible Assembly System to assemble batches of products generates an electronic event stream which is logged. The events in the event stream are typically at least
partially agnostic to the assembly process being performed, and therefore the events do not include tokens connecting the event to a specific final product instance.

This research is about evaluating different fault detection methods against simulated Flexible Automation System event streams.

This evaluation consists of model representation of interleaved symbolic time series, learning predictive patterns of the data, and validating new data against the predictions.
Symbolic time series in the context of this research means a sequence of discrete symbol and timestamp pairs.
In particular, representation of interleaved symbolic time series by a symbol-delay decomposition is described.
Symbol-delay decomposition projects the events to a multi-dimensional space with observed intervals
to previously encountered other symbols.

\section{Description of Method}

A proper literature review is a methodological, continuous process. The goal of the literature review is to accumulate a body of relevant existing knowledge
about the topic, categorized based on subtopics and keywords.
Ultimately the literature review can be presented in the article in a summary form to present the context of the research.
The collected references represent a focused area of the existing literature relevant to the object of research.

The review starts from discovery, discovering information sources and starting points of review. The literature review process progresses towards synthesis
where the relevant existing knowledge is synthesized together to form an understanding of the composite.

The discovery phase includes a listing of subtopics and keywords, to structure the gathered discovered information into a manageable form. The goal
of the discovery phase is to form questions about the existing knowledge and to find answers to them.

The synthesis is composed of the description of the existing knowledge and possible gaps related to the research.

\section{Discovery Phase}

\subsection{Keywords}

The keywords relevant for the research were collected from a set of articles deemed especially relevant for this research. A representative set of articles was
read and relevant keywords were picked from titles, abstracts and references. This set of keywords allows for directed browsing of relevant literature.

\begin{itemize}
 \item activity recognition
 \item alpha algorithm
 \item anomaly detection
 \item assembly process
 \item assembly system
 \item bidirectional LSTM
 \item causal reasoning
 \item Connectionist Temporal Classification
 \item continual prediction
 \item Convolutional Neural Network
 \item Deep Recurrent Neural Networks
 \item fault detection
 \item kalman filters
 \item keyhole plan recognition
 \item knowledge modelling
 \item labelling unsegmented sequence data
 \item learning Context Sensitive Languages
 \item Long Short-Term Memory (LSTM)
 \item petri nets
 \item plan recognition
 \item Stochastic Petri Nets
 \item Support Vector Machine classifier
 \item system verification
 \item transfer learning
 \item Weighted Unranked Tree Automata
\end{itemize}

The core questions about the existing literature are:
\begin{enumerate}
 \item What are the current best methods of industrial event log based anomaly detection or process mining?
 \item What methods are there to mathematically model logs generated by parallel processes, or shuffled languages?
 \item What are the relevant keywords and terms to describe this problem space?
\end{enumerate}

\section{Synthesis}

\bibliographystyle{IEEEtran}
\bibliography{LiteratureReview}

\end{document}
