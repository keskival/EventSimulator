\documentclass[a4paper,12pt]{article}
\usepackage[utf8]{inputenc}
\usepackage{url}
\usepackage{amsthm, amscd, amsfonts, amssymb, graphicx,tikz, color, environ}

%opening
\title{Literature Review: Simulating Flexible Assembly System Process Traces for the Purposes of Process Mining and Machine Learning – DRAFT}
\author{Tero Keski-Valkama}

\begin{document}

\maketitle

\smallskip
\noindent \textbf{Keywords.} Flexible Assembly System, Discrete Event Simulation, process mining, uncorrelated event streams, multi-modal representation

\tableofcontents

\newpage
\section{Introduction}
Flexible Assembly System is a modular and reconfigurable assembly and tooling workshop with a focus on small and medium-sized batches of varying products.
Flexible Assembly System was described formally by Donath and Graves \cite{donath1988flexible} as a system consisting of a set of products each with a specified volume
assembled on a workshop consisting of a fixed number of cells.

The assembly steps are performed in cells in parallel. Assemblies and components are transported to the cell where they are combined, tooled or inspected. Products
are transported out of the cell for the next assembly step in another cell, or out of the system as end products.
The work steps executed in the cells can be manual or automatic. There can be a central storage such as a shelf for storing components, the intermediate assemblies between the steps
and the end products waiting to be transported out of the system.

Flexible Assembly System work loads consist of small and medium sized batches where there can be some variation in the products based on customization and personalization.
The Flexible Assembly Systems are modular and often composed out of independent modules from different suppliers. The Flexible Assembly System event trace consists
of events received from all the separate modules of the system, and additional sensors and triggers added to the system in the integration phase or later.

Using a Flexible Assembly System to assemble batches of products generates an electronic event trace which is logged. The events in the event trace are typically at least
partially agnostic to the assembly process being performed, and therefore the events do not include tokens connecting the event to a specific final product instance.
The process traces formed by logs are therefore uncorrelated.

Simulating these kinds of event traces provides valuable material for learning systems which can be used to derive implicit, behavioristic assembly process models and
to detect deviances. Deviances which are not detected with current methods with alerts and fault codes are useful for detecting unexpected failure conditions, such as
human errors and cyberattacks.

It is in principle possible for an automatic system to detect features in the given material in an unsupervised fashion at least when a human is capable of doing so.
A multimodal approach is convenient in data representation when we need to explore human pattern recognition capabilities. In specific, data representation as audio
allows a human to understand the data melodies, rhythms and possible canon-like features.

\section{Description of Domain}

The topic under review is somewhat cross-discplinary relating to industrial assembly processes, business process modelling and learning systems.
Overall the following subtopics are recognized:
\begin{enumerate}
 \item Assembly process modelling and simulation
   \begin{enumerate}
     \item Fault modelling and simulation
   \end{enumerate}
 \item Mathematical analysis of log data
   \begin{enumerate}
     \item Analysing interleaved process traces
     \item Analysing delays and intervals
   \end{enumerate}
 \item Process mining
 \item Visualization of event logs
\end{enumerate}

\begin{figure}[ht]
\begin{center}
 \includegraphics[width=\textwidth]{./field.png}
 % field.png: 0x0 pixel, 300dpi, 0.00x0.00 cm, bb=
\end{center}
\caption{The mind map of the research topic}
\end{figure} 

\subsection{Keywords}

The keywords relevant for the research were collected from a set of articles deemed especially relevant for this research. A representative set of articles was
read and relevant keywords were picked from titles, abstracts and references. This set of keywords allows for directed browsing of relevant literature.

assembly process, assembly system, complex robotic system, compliant parts, dimensional quality, dimensional variation propagation, discrete event system,
Failure mode and effects analysis (FMEA), failure rates, failure records, failure report, fault detection, fault mode, fixture variation, interleaving,
leaf spring, multiple failure modes, Multi-Station Assembly Systems, outflow visualization,
part variation, petri nets, Process algebra Petri Net (PPN), process control, process failures, process FMEA, reduction of irregularities, reliability,
simulation, sonification, time-domain visualization, uncolored petri nets, uncorrelated event streams, variation propagation, visualization of sequences,
welding gun variation

\section{Literature Review}

Assembly systems, construction processes and business processes are often simulated using discrete event systems, for example in:
\cite{hlupic1998business,zhao2010efficient,kang2013active,rahnama2010fuzzy}. These simulations are used for process optimization\cite{sadeghi2008framework}, scheduling and
assembly line design. The models of these systems are conveniently described using Process algebra Petri Nets (PPNs) \cite{falkman2007specification}. Object Petri Nets
have been used in modelling hierarchical systems\cite{wu2015method}. For a lower abstraction level simulation IEC 61499 function block models\cite{tc652004wg6} have been used
for example in \cite{rooker2008modeling}.

The literature about faults and deviances in assembly processes mainly focus on the faults in the assembled products.
There are less academic publications about simulating process deviances in assembly processes such as unexpected process failures, human errors, and deadlocks.
Cyberattacks have received more attention, for example in \cite{wu2015method}.

The model of the simulated system is created in production line design \cite{bullinger}, or for existing systems by interviewing the system specialists \cite{montevechi2012using}.
The model is changed for each batch of different products being produced.

The fault types relevant in the context of this research are faults in the assembly system itself, not faults in the products being assembled.
The assembly system faults can be faults of specific assembly components, or systemic faults, such as deadlocks.
Different kinds of faults relevant for the Flexible Assembly Systems have been described for example in \cite{cong1997fault}.

A discrete event simulation output is a sequence of generated events,
and the internal system state between the events. The sequence of generated events is useful as a corpus for learning algorithms if it reflects the real conditions and
target phenomena well enough. In the case of anomaly detection, the simulation should be representative for the correct operation of the target system, and relevant process deviances
should in principle be observable in the output data model.

Flexible Assembly Systems assemble a batch of products in parallel.
Parallel processes create interleaved process traces. The set of all the interleaved process traces form a language. As this language consists of interleaved allowed sequences
it is a shuffle language\cite{berglund2011recognizing}.

Process mining is the field of inferring the underlying business processes based on observed events and transitions. The process mining methods can be used
to compare the supposed process model with the observed process model to detect deviances. Process mining methods have not been previously used for
Flexible Assembly System process modelling and verification using uncorrelated process traces.

The methods for describing process traces in the context of process mining are loosely based on the alpha algorithm\cite{van2004workflow} and as such, they expect process instances
to be identified in the activity events to deinterleave the process traces and to infer causal relations of activities.
``Event logs need to satisfy two main requirements: (a) events need to be ordered in time and (b) events need to be correlated
(i.e., each event needs to refer to a particular case)."\cite{van2011process}

There is very little academic literature about process mining of uncorrelated event streams. Uncorrelated event streams are important modelling targets when the event sources
are heterogeneous and not necessarily mapped to the specific steps in the known assembly process. This brings event sources such as emergency stop triggers, light curtains and
smart gateways into the process information system even if the log messages they trigger are not correlated to specific process instances.

Multi-modal representation of the event streams for the purpose of evaluating the human-observable presentation of features in the data is relevant in context of this research.
Having a human-detectable pattern in data where the same pattern is challenging to
detect with automatic means highlights a potential avenue for additional research.

If a pattern is evident for humans in the data representation, it is in principle discoverable by learning algorithms.

Different human sensory systems are tuned for particular types of patterns, for example music and speech
can be appreciated and recognized as audio, but the same signal patterns in visual representation are not recognized as well.
A multi-modal representation utilizes different sensory-cognitive capabilities to better facilitate detection of embedded patterns in the data.

There is a lot of literature about discovering structure in sequences and time series by using different visual representations and
projections \cite{hein2010recognition,misue2014chronoview}. Visualizations are also used in model-based reporting\cite{schuh2013ieee}.
Data representation as audio (Sonification \cite{refKra}) has been investigated for example in \cite{yeung1980pattern,kaper1999data}.

\newpage
\bibliographystyle{IEEEtran}
\bibliography{LiteratureReview}

\end{document}
